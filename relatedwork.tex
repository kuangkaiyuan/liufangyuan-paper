\section{Related work}
Obfuscation is a useful and cost effective technique and it does not require any special execution environment. Moreover it is believed to be more effective on Android system\cite{21,22}. Patrick Schulz in his work “Code Protection in Android”\cite{20} discusses some possible code obfuscation methods on the Android platform using identifier mangling, string obfuscation, dead code insertion, and self modifying code. Ghosh et al.\cite{23} have discussed a code obfuscation technique on the Android platform that aims at increasing the complexity of the control flow of the application so that it becomes tough for a reverse engineer to get the business logic performed by an Android application. Kundu has also worked on some obfuscation techniques like clone methods, reordering expressions and loops, changing the arrays and loop transformations\cite{24}.

There are various Android obfuscation tools available in the market, such as Proguad\cite{09}. But the current Android obfuscation tools seem to still lack the combination of complex control-flow and data-flow obfuscation techniques. In \cite{06,07,08} authors presents confusion scheme and algorithm of Android oriented software Java code, combined with the algorithm and improved insertion branch path and flattening the excess flow of control of these two kinds of control flow obfuscation method.

Junliang Shu et al. proposed SMOG\cite{10}, a comprehensive executable code obfuscation system to protect Android app. The obfuscation engine is at software vendor's side to conduct the obfuscation on the app's executable code, and then release the obfuscated app to the end-user along with an excution token. SMOG will also modify the code of DVM interpreter. Noor et al.\cite{26} present a protection scheme based on obfuscation, code modification and cryptographic protection that can effectively counter reverse engineering.

Vivek Bala et al.\cite{27} analyzed the need for potent control-flow based obfuscation so as to help protect Android apps. they also have described the design and implementation of three control-flow obfuscations for Android apps at the Dalvik bytecode level, which go beyond simp control-flow transformations used by exiting Android obfuscators. The register-reuse conflict problem raised by the Android runtime system has also been addressed by means of our type separation technique.
